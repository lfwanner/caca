%%%% Multiple Choice Questions

\element{basic-amdahl}{
 \begin{question}{amdahl1}
 If a program spends 60\% of its time executing multiplication instructions, how much faster do the multiplication instructions need to be for the program to run 2 times faster?
  \begin{choices}
   \correctchoice{6x faster.} \scoring{1}
   \wrongchoice{3x faster.}
   \wrongchoice{12x faster.}
   \wrongchoice{It is not possible to achieve the desired performance by optimizing only the multiplication instructions.}
   \wrongchoice{It is not possible to determine the answer with the given information.}
  \end{choices}
 \end{question}
}

\element{basic-short}{
  \begin{question}{reg-bits}
     How many bits are needed to address a register in the RISC-V register file?
    \begin{choiceshoriz}
      \correctchoice[5]{}
      \wrongchoice[16]{}
      \wrongchoice[32]{}
      \wrongchoice[6]{}
      \wrongchoice[8]{}
    \end{choiceshoriz}
  \end{question}
}

%%%% Simple Numeric Question

\element{num-fp}{
    \begin{questionmultx}{fpcv1}
      Convert the number 0xc0980000 from single-precision floating-point representation to decimal.
      \begin{center}
    \AMCnumericChoices{-4.75}{sign=true,vertical=true,base=10,digits=4,decimals=2,scoreexact=1}
    \end{center}
    \end{questionmultx}
}

%%%% Numeric Question with Randomization and FP calculations

\element{cache-dm-address}{

\FPeval{\blp}{round(random*4+1,0)}
\FPeval{\blw}{round(2^\blp,0)}
\FPeval{\ccp}{trunc(\blp+random*4+4,0)}
\FPeval{\ccb}{round(2^\ccp,0)}
\FPeval{\idx}{round(\ccp-\blp-2,0)}
\FPeval{\off}{round(\blp+2,0)}
\FPeval{\tag}{round(32-\off-\idx,0)}

\begin{questionmultx}{indexsize} 
    Consider an architecture with 32-bit words and a direct-mapped L1 data cache with \ccb{} bytes capacity. If the cache has blocks of \blw{} words, how many bits are required for the index field in the address division?
    \begin{center}
        \AMCnumericChoices{\idx}{sign=false,vertical=false,nozero=true,base=10,digits=1,scoreexact=1}
    \end{center}
\end{questionmultx}

}

%%%% Numeric Question with Python

\begin{pycode}

def mc_div(ddend, dsor, b, cycle):
    dsor <<= b
    quotient = 0
    remainder = ddend
    if cycle == 0 :
        return  {'Quotient':quotient, 
                 'Remainder':remainder, 
                 'dsor':dsor}
    for i in range(1,b+2): 
        remainder = remainder - dsor
        if remainder < 0:
            remainder = remainder + dsor
            quotient <<= 1
        else :
            quotient <<= 1
            quotient += 1
        dsor >>= 1
        if i == cycle:
            return  {'Quotient':quotient, 
                     'Remainder':remainder, 
                     'dsor':dsor}

def div_q(reg, ddend, dsor, b, cycle, pts):
    ans = mc_div(ddend, dsor, b, cycle)[reg]
    bsize = b * 2 if (reg == 'dsor') else b
    print("""
    \\begin{questionmultx}{Division%s%d}
    \\\\%s, Cycle %d
    \\begin{center}
    \\AMCnumericChoices{%d}{sign=false,vertical=true,base=2,digits=%d,scoreexact=%f}
    \\end{center}
    \\end{questionmultx}
    """ %(reg, cycle, reg, cycle, ans, bsize, pts))

\end{pycode}


\element{divider}{

\pyc{import random; dividend=random.randint(11, 14); divisor = random.randint(3, 5);}

For the next question, considering a multi-cycle divider, divide \py{dividend} by \py{divisor}, indicating the value of the specified register at the end of the clock cycle. Assume that the inputs and results are represented with 4 bits, and give your answer in binary.

\pyc{div_q('Remainder', dividend, divisor, 4, 3, 1)}

}


%%%% Open Question

\element{asm}{
\begin{question}{open-asm}
 Implement the \texttt{int strlen(const char* str)} function in RISC-V assembly, following all necessary ABI conventions and without using pseudo-instructions. The function should return the length of the string.
\AMCOpen{lineheight=0.6cm,lines=5}{
\wrongchoice[0]{0}\scoring{0}
\wrongchoice[1]{1}\scoring{1}
\wrongchoice[2]{2}\scoring{2}
\wrongchoice[3]{3}\scoring{3}
\wrongchoice[4]{4}\scoring{4}
\correctchoice[5]{5}\scoring{5}}
\end{question}
}

