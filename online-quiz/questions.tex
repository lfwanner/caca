\element{fp}{
     \begin{questionmultx}{fp1}

     How is the number 13.75 represented in half-precision floating-point notation?
     \begin{center}
     \AMCnumericChoices{19168}{sign=false,vertical=true,base=2,digits=16,scoreexact=2}
     \end{center}
     \end{questionmultx}
 }

 \element{fp}{
     \begin{questionmultx}{fp2}
     How is the number 22.5 represented in half-precision floating-point notation?
     \begin{center}
     \AMCnumericChoices{19872}{sign=false,vertical=true,base=2,digits=16,scoreexact=2}
     \end{center}
     \end{questionmultx}
 }

 \element{fp}{
     \begin{questionmultx}{fp3}
     How is the number 18.75 represented in half-precision floating-point notation?
     \begin{center}
     \AMCnumericChoices{19632}{sign=false,vertical=true,base=2,digits=16,scoreexact=2}
     \end{center}
     \end{questionmultx}
 }

 \element{fp}{
     \begin{questionmultx}{fp4}
     How is the number 26.5 represented in half-precision floating-point notation?
     \begin{center}
     \AMCnumericChoices{20128}{sign=false,vertical=true,base=2,digits=16,scoreexact=2}
     \end{center}
     \end{questionmultx}
 }

 \element{fp}{
     \begin{questionmultx}{fp5}
     How is the number 26.25 represented in half-precision floating-point notation?
     \begin{center}
     \AMCnumericChoices{20112}{sign=false,vertical=true,base=2,digits=16,scoreexact=2}
     \end{center}
     \end{questionmultx}
 }

 \element{fp-repr}{


 \noindent Consider a half-precision floating-point representation using 16 bits with the following characteristics:
 \begin{multicols}{2}
 \begin{itemize}
   \item Sign Bit (Bit 15)
   \item 5-bit Exponent (Bits 14--10)
   \item 10-bit Fractional Part (Bits 9--0)
   \item Exponent bias equal to 15
 \end{itemize}
\end{multicols}

\noindent This representation follows all the conventions of the IEEE 754 floating-point representations, including normalization, value conversion, representation of infinity and NaN, denormal numbers, etc. 

 \shufflegroup{fp}
 \insertgroup[1]{fp}

 }
